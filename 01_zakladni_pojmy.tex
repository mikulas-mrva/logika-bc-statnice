\documentclass[12pt,a4paper]{article}
\usepackage{mathrsfs}
\usepackage{amssymb}
\usepackage{amsmath}
\usepackage{amsfonts}
\usepackage{longtable}
\usepackage{paralist}
\usepackage{lineno}
\usepackage{verbatim}
\usepackage[utf8x]{inputenc}
\usepackage{csquotes}
\pagestyle{plain}
\headheight 14.5pt
%\fancyhead[R]{\leftmark}
%\fancyfoot{}
%\fancyfoot[C]{\thepage}


\newtheorem{theorem}{Theorem}[section]
\newtheorem{Claim}[theorem]{Claim}
\newtheorem{definition}[theorem]{Definition}
\newtheorem{Cor}[theorem]{Corollary}
\newtheorem{Fact}[theorem]{Fact}
\newtheorem{lemma}[theorem]{Lemma}
\newtheorem{sublemma}[theorem]{Sublemma}
\newtheorem{ex}[theorem]{Example}
\newtheorem{remark}[theorem]{Remark}
\newtheorem{obs}[theorem]{Observation}
\newtheorem{que}[theorem]{Question}
\newtheorem{conjecture}[theorem]{Conjecture}

\renewcommand{\theequation}{\thesection.\arabic{equation}}

\newenvironment{proof}
{\noindent \textit{Proof.}}
{\hspace*{\fill} $\Box$}

\newcommand{\toch}{\fbox{\small {\bf ??}}}
\newcommand{\bt}[1]{{\underset{\widetilde{}}{#1}}}
\newcommand{\trcl}[1]{\ensuremath{\mathrm{trcl}(\{#1\})}}
\newcommand{\cf}[1]{\ensuremath{\mathrm{cf}(#1)}}
\newcommand{\cl}[1]{\ensuremath{\mathrm{cl}}(#1)}
\newcommand{\ord}[1]{\ensuremath{\mathrm{ORD}}(#1)}
\newcommand{\dom}[1]{\ensuremath{\mathrm{dom}}(#1)}
\newcommand{\rng}[1]{\ensuremath{\mathrm{rng}}(#1)}
\newcommand{\power}[1]{\ensuremath{\mathscr{P}} (#1)}
\newcommand{\set}[2]{\ensuremath{\{#1 \,|\, #2 \}}}
\newcommand{\seq}[2]{\ensuremath{\langle #1 \,|\, #2 \rangle}}
\newcommand{\singl}[1]{\ensuremath{\{#1\}}}
\newcommand{\pair}[2]{\ensuremath{\{ #1, #2 \}}}
\newcommand{\restr}[2]{\ensuremath{#1 \! \upharpoonright \! #2}}
\renewcommand{\iff}{\leftrightarrow}
\newcommand{\Iff}{\Leftrightarrow}
\newcommand{\el}{\prec}
\newcommand{\iso}{\cong}
\newcommand{\sub}{\subseteq}
\newcommand{\super}{\supseteq}
\newcommand{\la}{\langle}
\newcommand{\ra}{\rangle}
\newcommand{\embed}{\rightarrow}
\newcommand{\mc}{\mathcal}
\newcommand{\supr}[1]{\mathrm{sup}\,#1}
\newcommand{\then}{\rightarrow}
\newcommand{\conc}{^{\smallfrown}}
\newcommand{\bb}{\mathbb}
\newcommand{\supp}[1]{\mathrm{supp}(#1)}
\newcommand{\beq}{\begin{equation}}
\newcommand{\eeq}{\end{equation}}
\newcommand{\brm}{\begin{remark}\begin{rm}}
\newcommand{\erm}{\end{rm}\end{remark}}
\newcommand{\mx}{\mathrm}
\newcommand{\bce}{\begin{compactenum}}
\newcommand{\ece}{\end{compactenum}}
\newcommand{\op}[2]{\la #1, #2 \ra}
\newcommand{\treq}{\trianglelefteq}
\newcommand{\et}{\mathrel{\&}}

\newcommand\defeq{\mathrel{\overset{\makebox[0pt]{\mbox{\normalfont\tiny\sffamily def}}}{=}}}


\begin{document}

Pozor na rozlišení formule / výrokové fomrule, atomy / výrokové atomy, etc...

\begin{definition}{(Množina všech atomů)}\\
Množnou všech výrokových atomů, značenou $\mathsf{At}$, nazýváme nekonečnou množinu všech proměnných
\begin{equation}
\mathsf{At} \defeq \{p_n : n \in \sf{N}\}
\end{equation}
\end{definition}

\begin{definition}{(Výrokové spojky)}
$\neg, \et, \lor, \then$ definovat netřeba.
\end{definition}

\begin{definition}{(Literál)}\\
\bce[(i)]
\item Atom je \emph{literál}.
\item Negace atomu je \emph{literál}.
\ece
\end{definition}

\begin{definition}{(Klauzule)}
Formule $\varphi$ je klauzule, když je ve tvaru
\begin{equation}
\varphi = \psi_1 \lor \psi_2 \lor \ldots \lor \psi_n 
\end{equation}
pro některé $n \in \sf{N}$.
\end{definition}

\begin{definition}{(Množina všech fomrulí)}\\
Rekurzivne podle definice spojek $\ldots$. Značime $\mathsf{Fle}$
\end{definition}

\begin{definition}{(Pravdivostní ohodnocení)}\\
$mathsf{At}$ je množina všech atomů. Pravdivostní ohodnocení je libovolná funkce v:
\begin{equation}
v : \mathsf{At} \then \{0, 1\}
\end{equation}
Jeho rozšíření na $\bar{v}: \mathsf{Fle}$ 
\end{definition}

\begin{definition}{(splnitelná (výroková) formule)}\\
Výroková formule $\varphi$ je splnitelná, jestliže existuje pravdivostní ohodnocení v takové, že $v(\varphi) = 1$
\end{definition}

\begin{definition}{(splnitelná množina (výrokových) formulí)}\\
Množina výrokových formulí $T$ je splnitelná, jestliže existuje pravdivostní ohodnocení v takové, že $v(\psi) = 1\mbox{ pro každou }\psi \in T$
\end{definition}

\begin{definition}{(vyplývání)}\\
Výroková formule $\varphi$ je (tautologickým) důsledkem množiny výrokových formulí T, jestliže $\varphi$ je splněna
každým pravdivostním ohodnocením, které splňuje všechny formule z T. Píšeme $T \models \varphi$
\begin{equation}
T \models \varphi iff \forall v(\forall\psi \in T(v(\psi) = 1) \then v(\varphi) = 1)
\end{equation}
\end{definition}

\begin{definition}{(Tautologie)}\\
(Výroková) Formule $\varphi$ je (výroková) tautologie, jestliže $v(\varphi) = 1$ pro každé pravdivostní ohodnocení v.\
Píšeme $\models \varphi$, nebo $\emptyset \models \varphi$, tautologie vyplývá z prázdné množiny (výrokových) formulí.
\end{definition}

\subsubsection{Predikáová logika}

\begin{definition}{(Realizace)}\\
Realizací jazyka rozumíme algebraickou strukturu složenou z:
\bce[(i)]
\item  neprázdné množiny $M$ zvané univerzum
\item pro každý funkční symbol $f$ arity $n$ je dáno zobrazení $f: M^n \rightarrow M$
\item pro každý predikátový symbol $R$ arity $n$ kromě rovnosti je dána relace $R \subseteq M^n$
\item relace $=$ představuje rovnost objektů z $M$
\ece
\end{definition}

\begin{definition}{(Ohodnocení proměnných)}\\
Ohodnocením proměnných nazveme zobrazení z množiny e všech proměnných do univerza M dané realizace jazyka predikátové logiky.
Ohodnocení proměnné $x$ prvkem $m$ v rámci všech ohodnocení proměnných $e$ značíme $e(x) = m$ nebo též $e(x/m)$.
\end{definition}

\begin{definition}{(Pravdivostní hodnota formule)}\\
Indukcí definujeme, co je to pravdivostní hodnota formule $\varphi$v realizaci M při ohodnocení e:
\bce[(i)]
\item je-li $\varphi$atomická formule tvaru $R(t_1, t_2, \dots, t_n)$, kde $R$ je predikátový symbol arity $n$ a $t_1,\ldots, t_n$ jsou termy, pak $\varphi$je pravdivá právě tehdy, když $(e(t_1), e(t_2), \dots, e(t_n)) \in R$.
\item je-li $\varphi$ atomická formule tvaru $t1 = t2$, kde $t1$ a $t2$ jsou termy, pak $\varphi$ je pravdivá, právě když $e(t1)$ je tentýž prvek jako $e(t2)$.
\item je-li $\varphi$ tvaru $\neg \psi$, kde $\psi$,je formule jazyka, pak $\varphi$ je pravdivá, právě když $\psi$ není pravdivá formule, analogicky ostatní spojky.
\item je-li $\varphi$ tvaru $(\forall x) \psi$, kde $\psi$ je formule jazyka, pak $\varphi$ je pravdivá právě tehdy, když pro každý prvek $m \in M$ je $\psi[e(x/m)]$ pravdivé.
\item je-li $\varphi$ tvaru $(\exists x) \psi$, kde $\psi$ je formule jazyka, pak $\varphi$ je pravdivá právě když existuje $m \in M$ tak, že $Q[e(x/m)]$ je pravdivé.
\ece
\end{definition}

\begin{definition}{(logicky platná formule)}\\
Formule $\varphi$ jazyka $\mathscr{L}$ se nazývá logicky platná, jestliže je platná ve všech realizacích.
\end{definition}

\begin{definition}{(disjunktivní a konjunktivní normální forma)}\\
\bce[(i)]
\item Řekneme že formule je v \emph{disjunktivní normální formě}, jestliže je to disjunkce konjunkcí literálů.
\item Řekneme že formule je v \emph{konjunktivní normální formě}, jestliže je to konjunkce disjunkcí literálů.
\ece
\end{definition}

\begin{definition}{(prenexní normální forma)}\\
Řekneme že formule $\varphi$ je v \emph{prenexní normální formě}, jestliže je v jejím zápisu všechny kvantifikátory předcházejí všechny logické spojky.
\end{definition}

\begin{theorem}{(Věta o dedukci)}\\
Nechť $\Gamma$ je množina formulí a $\varphi$ a $\psi$ jsou formula takové, že $\Gamma, \psi \vdash \varphi$.\\ Pak $\Gamma \vdash \psi \then \varphi$.
\end{theorem}

\begin{definition}{(axiomatická teorie)}\\
Axiomatická teorie je dvojice jazyka $L$ a $T$, množiny sentencí v $L$. Místo $\langle L, T \rangle$ píšeme obvykle $T$.\footnote{Občas jsou axiomy ve skutečnosti univerzální uzávěry formulí v $T$.}
\end{definition}

\begin{definition}{(Model)}\\
Struktura $D$ pro jazyk $L$ je model teorie $\langle L,T \rangle$, jestliže v $D$ platí všechny prvky množiny $T$.
\end{definition}

\begin{definition}{(rekurzívně spočetné a rekurzívní množiny a Postova věta)}\\
$\ldots$
\end{definition}

\begin{definition}{(pojem funkce)}\\
$\ldots$
\end{definition}

\begin{definition}{(relace a vlastnosti relací)}\\
$\ldots$
\end{definition}

\begin{definition}{(grupa)}\\
$\ldots$
\end{definition}

\begin{definition}{(okruh)}\\
$\ldots$
\end{definition}

\begin{definition}{(ordinální a kardinální čísla)}\\
$\ldots$
\end{definition}

\end{document}